\documentclass[12pt]{report}
\usepackage{graphicx}
\usepackage{float}
\usepackage{lettrine}
\usepackage{caption}
\usepackage{url}
\usepackage[margin=1in]{geometry}
\usepackage{amsmath}
\usepackage{algorithm}
\usepackage[noend]{algpseudocode}
\usepackage{amssymb}

\newcommand\blfootnote[1]{%
  \begingroup
  \renewcommand\thefootnote{}\footnote{#1}%
  \addtocounter{footnote}{-1}%
  \endgroup
}

\title{Verification of GCD Circuits Using Algebraic Geometry}
\author{Jaden Simon - simonjaden223@gmail.com \\ \and
	   Daniel Humeniuk - d.humeniuk@utah.edu}

	   
\begin{document}

\maketitle

\section{Introduction}

The purpose of this report is to discuss our findings and results of our formal verification project for ECE 5745. Our original proposal was centered around verifying a 3-bit Elliptical Curve Arithmetic Unit (ECAU) circuit as a proof of concept. After some studying, our project led us to investigate the 3-bit greatest common divisor (GCD) circuit which would be required in a 3-bit ECAU. Our main point of investigation was to determine if verification of a 3-bit GCD circuit is trivial or nontrivial. Once this question could be answered, we could later explore if the verification could be easily transposed onto an $n$-bit GCD circuit.

This report will detail our findings as well as identified areas of further interest and exploration.

\section{Experiments and Test Setup}

Our first starting point was designing a circuit that computed a 3-bit GCD. This circuit can be seen in Figure \ref{fig:gcd}. Once this circuit was created, it took some refining to ensure that it would act as expected. The GCD as implemented here would require at most $n$ clock cycles for an $n$-bit circuit. 

\begin{figure}
\includegraphics[scale=0.35]{images/gcd_page.png}
\caption{Binary GCD Algorithm for $3$ bits}
\label{fig:gcd}
\end{figure}

\section{Findings}

\section{Conclusion}

\bibliographystyle{IEEEtran}

%\bibliography{IEEEabrv,bib/ref}

\end{document}